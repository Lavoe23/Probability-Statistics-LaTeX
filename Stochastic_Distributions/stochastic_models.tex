\documentclass[a4paper,11pt]{scrartcl}
\usepackage[utf8]{inputenc}
\usepackage[spanish]{babel}
\usepackage{amsmath}
\usepackage{amsfonts}
\usepackage{graphicx}
\usepackage{geometry}
\geometry{left=2.5cm, right=2.5cm, top=2.5cm, bottom=2.5cm}

% --- METADATA DEL DOCUMENTO ---
\title{Modelado Estocástico y Distribuciones de Probabilidad}
\subtitle{Informe Técnico: Análisis de Variables Aleatorias Continuas y Discretas}
\author{Camilo Aranda}
\date{\today}

\begin{document}

\maketitle

\begin{abstract}
\noindent \textbf{Resumen:} Este documento presenta la resolución analítica de problemas complejos de probabilidad y control de calidad. Se abordan modelos basados en la \textbf{Distribución Normal} para inferencia de intervalos de confianza, la \textbf{Distribución Exponencial} para análisis de fiabilidad de componentes y el uso del \textbf{Teorema de la Probabilidad Total} en procesos estocásticos mixtos. El análisis incluye estandarización de variables ($Z$-scores) y cálculo de esperanzas condicionales.
\end{abstract}

\vspace{0.5cm}
\hrule
\vspace{0.5cm}

%-------------------------------------------------------------------------

\section*{Caso de Estudio 1: Control de Calidad en Capacitores}

Sea $X$ la resistencia (en megohm) de un capacitor, modelada como una variable aleatoria normal con media $\mu=800$ y desviación estándar $\sigma=200$:

$$
X \sim \mathcal{N}(800,\,200^2).
$$

Para los cálculos se utiliza la variable estandarizada $Z = \frac{X-\mu}{\sigma} \sim \mathcal{N}(0,1)$.

\subsection*{a) Proporción de cumplimiento de especificaciones}

Se requiere determinar la probabilidad de que la resistencia esté en el rango $[900, 1000]$:

$$
P(900 \le X \le 1000)
= P\left(\frac{900-800}{200} \le Z \le \frac{1000-800}{200}\right)
$$
$$
= P(0.5 \le Z \le 1)
= \Phi(1) - \Phi(0.5).
$$

Consultando la función de distribución acumulada (CDF):
$$
\Phi(1)=0.841, \qquad \Phi(0.5)=0.691
$$
$$
P(900 \le X \le 1000)=0.841-0.691=0.150.
$$

\textbf{Resultado:} La proporción de capacitores que cumple con la especificación técnica es del \textbf{15.0\%}.

\subsection*{b) Intervalo de confianza del 95\%}

Buscamos el intervalo simétrico centrado en la media que contenga el $95\%$ de la población. Esto corresponde a:
$$
\mu \pm z_{0.975}\sigma, \quad \text{donde } z_{0.975} \approx 1.96.
$$

$$
800 \pm 1.96(200)=800 \pm 392 \Rightarrow [408,\ 1192].
$$

\textbf{Resultado:} El 95\% de los capacitores tiene una resistencia entre $408$ y $1192$ megohm.

\subsection*{c) Probabilidad en muestreo secuencial (Distribución Binomial Negativa)}

Sea $N$ el número de capacitores revisados. Se busca la probabilidad de revisar $n=5$ para encontrar $k=2$ éxitos (cumplimiento de norma), sabiendo que la probabilidad de éxito es $p=0.150$.

Esto implica: 1 éxito en los primeros 4 intentos y éxito en el 5º intento.
$$
P(N=5)=\binom{4}{1}p^2 (1-p)^3.
$$
$$
P(N=5) = 4(0.150)^2(0.850)^3 \approx 0.055.
$$

\textbf{Resultado:} La probabilidad es aproximadamente \textbf{5.5\%}.

%-------------------------------------------------------------------------
\newpage
\section*{Caso de Estudio 2: Métricas de Retención en Marketing}

Sea $X$ el porcentaje de retención de un comercial, modelado como $X \sim \mathcal{N}(70,\,10^2)$.

\subsection*{a) Probabilidad de éxito comercial}
Criterio de éxito: $X > 80\%$.
$$
P(X>80)=P(Z>1)=1-\Phi(1)=1-0.841=0.159.
$$
\textbf{Conclusión:} Solo el \textbf{15.9\%} de los comerciales alcanza el estatus de éxito.

\subsection*{b) Estimación poblacional (Esperanza)}
Para una población de $N=2500$ niños, calculamos la proporción esperada en el rango $[60, 80]$:
$$
P(60 \le X \le 80) = P(-1 \le Z \le 1) = \Phi(1)-\Phi(-1) = 0.682.
$$
$$
\text{Esperanza} = 2500 \cdot 0.682 = 1705.
$$
\textbf{Conclusión:} Se espera que 1705 niños presenten una retención media.

\subsection*{c) Umbral de baja atención (Percentil 2.5)}
Buscamos el valor $l$ tal que $P(X \le l)=0.025$. El $z$-score asociado es $-1.96$.
$$
l = \mu + z\sigma = 70 + (-1.96)(10) = 50.4.
$$
\textbf{Conclusión:} Cualquier retención bajo \textbf{50.4\%} se clasifica como déficit de atención.

%-------------------------------------------------------------------------
\newpage
\section*{Caso de Estudio 3: Fiabilidad en Semiconductores}

Sea $X$ el espesor de una película fotoprotectora. Dado que el intervalo $[8.6, 13.4]$ contiene al $95.44\%$ de los datos (lo que corresponde a $\mu \pm 2\sigma$), deducimos los parámetros:

\subsection*{a) Inferencia de Parámetros}
$$
\begin{cases}
\mu - 2\sigma = 8.6 \\
\mu + 2\sigma = 13.4
\end{cases}
\implies 2\mu = 22 \implies \mu = 11.0, \quad 4\sigma = 4.8 \implies \sigma = 1.2.
$$
Varianza: $\sigma^2 = 1.44$.

\subsection*{b) Análisis Geométrico de Fallos}
Probabilidad de espesor óptimo ($9 \le X \le 13$):
$$
p = P\left(\frac{9-11}{1.2} \le Z \le \frac{13-11}{1.2}\right) = P(-1.67 \le Z \le 1.67) \approx 0.905.
$$

Modelamos el número de revisiones $N$ hasta el primer éxito como una Distribución Geométrica. Buscamos $P(N \le 3)$:
$$
P(N \le 3) = 1 - (1-p)^3 = 1 - (0.095)^3 \approx 0.999.
$$
\textbf{Conclusión:} Es casi seguro (99.9\%) que se encontrará un semiconductor óptimo en los primeros 3 intentos.

%-------------------------------------------------------------------------
\newpage
\section*{Caso de Estudio 4: Fiabilidad Conjunta (Hardware)}

Sistema con dos variables independientes:
\begin{itemize}
    \item $X \sim \text{Exp}(1/6)$: Tiempo de vida (años).
    \item $Y \sim \mathcal{N}(2.8, 0.2^2)$: Velocidad (GHz).
\end{itemize}

\subsection*{a) Probabilidad de Defecto Total}
Un componente falla si $X < 0.5$ o $Y < 2.4$.
$$
P(\text{Falla}) = 1 - P(X \ge 0.5)P(Y \ge 2.4).
$$
$$
P(X \ge 0.5) = e^{-0.5/6} \approx 0.920.
$$
$$
P(Y \ge 2.4) = P(Z \ge -2) \approx 0.977.
$$
$$
P(\text{Falla}) = 1 - (0.920 \cdot 0.977) \approx 0.101.
$$
\textbf{Riesgo:} Existe un \textbf{10.1\%} de probabilidad de fallo en el componente.

\subsection*{b) Diferencia de Rendimiento (Combinación Lineal de Normales)}
Sean $Y_1, Y_2$ independientes. Definimos $D = Y_1 - Y_2 \sim \mathcal{N}(0, 2\sigma^2)$.
$$
\sigma_D = \sqrt{0.04 + 0.04} \approx 0.283.
$$
$$
P(D > 0.5) = P(Z > 1.768) \approx 0.039.
$$
%-------------------------------------------------------------------------
\newpage
\section*{Caso de Estudio 5: Resistencia de Materiales (Probabilidad Condicional)}

Se analizan dos variables independientes en hormigón:
\begin{itemize}
    \item $X$ (Compresión) $\sim \mathcal{N}(33, 5.016^2)$.
    \item $Y$ (Tracción) $\sim \mathcal{N}(2.6, 0.468^2)$.
\end{itemize}

\subsection*{a) Probabilidad Condicional}
Se busca $P(X < 34.12 \mid X > 30.5)$. Aplicando la definición de Bayes:
$$
P(X < 34.12 \mid X > 30.5) = \frac{P(30.5 < X < 34.12)}{P(X > 30.5)}.
$$

Estandarizando los límites a valores $Z$:
$$
P(-0.498 < Z < 0.223) = \Phi(0.223) - \Phi(-0.498) \approx 0.279.
$$
El denominador (Probabilidad total del condicional):
$$
P(Z > -0.498) \approx 0.691.
$$

$$
\text{Resultado} = \frac{0.279}{0.691} \approx 0.404.
$$
\textbf{Conclusión:} Existe un \textbf{40.4\%} de probabilidad bajo la condición dada.

\subsection*{b) Seguridad Conjunta (Independencia)}
Condición de seguridad óptima: $A=\{24.7 < X < 41.3\}$ y $B=\{Y > 2.85\}$.
Al ser independientes:
$$
P(\text{Óptima}) = P(A) \cdot P(B).
$$
$$
P(A) \approx 0.902, \qquad P(B) \approx 0.297.
$$
$$
P(\text{Óptima}) = 0.902 \cdot 0.297 \approx 0.268.
$$
\textbf{Conclusión:} La probabilidad de seguridad óptima es del \textbf{26.8\%}.
%-------------------------------------------------------------------------
\newpage
\section*{Caso de Estudio 6: Proceso Estocástico Mixto (Dado + Moneda)}

Experimento en dos etapas:
1. Lanzar un dado ($D \in \{1..6\}$).
2. Lanzar una moneda $D$ veces. Sea $X$ el número de caras.

\subsection*{a) Función de Masa de Probabilidad (Teorema de Probabilidad Total)}
La distribución marginal de $X$ es una mezcla de Binomiales:
$$
P(X=k) = \sum_{n=1}^{6} P(X=k \mid D=n) P(D=n).
$$
$$
P(X=k) = \frac{1}{6} \sum_{n=k}^{6} \binom{n}{k} (0.5)^n.
$$

\subsection*{b) Esperanza Condicional Iterada}
Por la ley de la esperanza total:
$$
E[X] = E[E[X|D]] = E[D/2] = \frac{E[D]}{2} = \frac{3.5}{2} = 1.75.
$$

\textbf{Análisis de Cola:}
Probabilidad de superar la esperanza ($P(X \ge 2)$):
$$
P(X \ge 2) = 1 - [P(X=0) + P(X=1)] \approx 1 - (0.164 + 0.313) = 0.523.
$$

\end{document}
